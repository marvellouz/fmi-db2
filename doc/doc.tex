\documentclass[a4paper,10pt, leqno]{article}
\usepackage[utf8]{inputenc}
\usepackage[bulgarian]{babel}
\newcommand{\HRule}{\rule{\linewidth}{0.5mm}}
\usepackage{amsmath}
\usepackage[unicode]{hyperref}
\hypersetup{
  colorlinks=true,
  linkcolor=black,
  pdftitle={Проект по ПФНРсСУБД2},
  pdfauthor={Валентина Динкова, Емил Станчев},
  pdfsubject={Бази от данни},
  pdfcreator={Емил Станчев, Валентина Динкова}
}
\usepackage{yfonts}
\usepackage{amsthm}
\usepackage {amssymb}
\usepackage{graphicx}
% for source code inclusion
\usepackage{listings}
\renewcommand{\lstlistingname}{\bfseries Файл}
\lstset{ language=Matlab, caption=\lstname, numbers=left, numberstyle=\tiny}
\lstset{keywordstyle=\bfseries}
\lstset{commentstyle=\color{darkgray}}
%\lstset{frame=T}
\newcommand{\attr}[1] {\texttt{#1}}
\newcommand{\enti}[1] {\textbf{#1}}
\newcommand{\re}[1] {\texttt{#1}}

\begin{document}

% TITLE PAGE
\begin{titlepage}
  \begin{center}
    \textsc{СУ ``Св. Климент Охридски''\\
    Факултет по математика и информатика
    }\\[5cm]

    \textsc{\large Курсов проект по \\
	   	   Проектиране на физическо ниво и реализация със СУБД II
    }\\[0.5cm]
    \HRule \\[0.4cm]
    { \Large \bfseries База от данни за система за електронно обучение\\ \\[0.4cm]
    }\\[0.4cm]
    \HRule \\[6cm]
    \begin{minipage}{0.49\textwidth}
      \begin{flushleft} \large
	\emph{Автори:}\\
	Валентина Динкова,\\
	{\small ф.н.71112}\\
	Емил Станчев\\
	{\small ф.н.71100}\\
      \end{flushleft}
    \end{minipage}
    \begin{minipage}{0.49\textwidth}
      \begin{flushright} \large
	\emph{Ръководители:} \\
	доц. В. Димитров\\
	ас. Р. Горанова
      \end{flushright}
    \end{minipage}
    \vfill
    {\large \today}
  \end{center}
\end{titlepage}

\tableofcontents
\newpage

	\section{Описание}
	  Проектът представлява модел и примерна реализация на база от данни, предназначена за приложение за електронно обучение.
	  В него учителите качват материали за различни курсове, студентите обсъждат във форум различни теми, свързани с курсовете,
	  както и	административни въпроси. Учителите могат да качват задания с определен краен срок, за които студентите получават оценка.
	  Студентите могат да дават оценка на преподавателите в даден курс. Всички потребители се идентифицират с парола и email адрес.

	\section{Множества същности}
	Най-важните множества същности са:
	\begin{description}
	    \item[\enti{Course}]
	      Курс с име \attr{name} за дадена година. Може да има курсове с еднакви имена в различни години, затова ключът се състои
	      от името и годината на курса. Освен това има опционална парола \attr{password} за записване на курса. Пази се и броят на записаните
	      в курса студенти \attr{numEnrolled}. Всеки курс има един титуляр \attr{titular} и други учители \re{OtherTeachers}.
	      Всеки курс има категория \enti{Category}.
	      Всеки студент \enti{StudentProfile}, записан чрез \re{Enrolled} получава оценка за дадения курс \re{CourseGrade}, която има стойност
	      \attr{value}.
	    \item[\enti{User}]
	      Потребител, който има парола \attr{password}, \attr{email}, и имена \attr{first name}, \attr{last name}.
	      Всеки потребител има поне едно от \enti{StudentProfile} и \\
	      \enti{TeacherProfile}, които са слаби множества същности.
	    \item[\enti{Assignment}]
	      Задание, което има краен срок \attr{deadline}, максимален брой точки, които дава заданието \attr{max points}, заглавие \attr{title},
	      описание \attr{description}, уникален номер \attr{number} и дата на създаване \attr{created at}. Авторът на всеки \enti{Assignment}
	      е \enti{TeacherProfile} на някой \enti{User}. Всеки \enti{Assignment} може да има чрез \re{Attached} прикачени файлове \enti{File},
	      които имат име \attr{name}
	      и път \attr{path}, който е ключ за файла. Всяко задание принадлежи на даден курс.
	    \item[\enti{Resource}]
	      Ресурс, който принадлежи на даден \enti{Course}, напр. лекция, публикация и др. 
	      Ресурсът може да има прикачени файлове.
	    \item[\enti{ForumThread}]
	      Тема в дискусионния форум на даден курс, която има заглавие \attr{title} и тяло \attr{body}. Всяка тема може да има
	      отговори \enti{ForumReply}. Всички теми или отговори имат автори, които са \enti{User}.
	    \item[\enti{ForumThread}]
	    \item[\enti{Notification}]
	      Известие за настъпило събитие, като например прибавяне или изтриване на \enti{Assignment}, \enti{ForumReply} и др.
	\end{description}
\end{document}
